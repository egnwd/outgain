\documentclass[a4paper, 6pt]{article}
\usepackage[margin=50pt]{geometry}
\usepackage[T1]{fontenc}
\usepackage{titling}
\usepackage{graphicx}
\usepackage{multicol}
\pagestyle{empty}
\title{Law Case Study}
\author{Web Apps Group 08 \\ Elliot Greenwood, Paul Li\'etar, Michael Radigan \& Jamie Lane}
\date{}
\setlength{\droptitle}{-5em}
\begin{document}
\maketitle
\section{Question 1}
\subsection{Conflicts with freedom 0}
\begin{itemize}
    \item ``you may only use the tools included in the Software to build the Singularity system, or build applications that will run on Singularity'' conflicts directly with the ability to run the tools as you wish, for any purpose.
    \item The license prohibits the user from running the code for the ``creation or use of commercial products or any other activity which purpose is to procure a commercial gain to you or others''. Again, this conficts directly with freedom 0.
    \item Section 3: not being able to use any ``program... created through the use of the software... in the same manner as a comercially released product''. This conflicts with the the user's freedom to run for any purpose.
    \item Section 3: ``with data that has not been sufficiently backed up''. The fact that certain programs are not able to be ran using your own data if it is not backed up is definitely in disagreement with freedom 0.
    \item 
\end{itemize}
\subsection{Conflicts with freedom 1}
\begin{itemize}    
    \item ``You may not use... any derivative works in any form for commercial purposes'' is in conflict with the freedom to ``change it so it does your computing as you wish'' in freeedom 1.
    \item Freedom 1 requires that the user may ``change it... as you wish'', however MSR-LA allows derivative works to be used ``solely for non-commercial academic purposes''.
    \item Section 2: The MSR-LA does not permit you to ``modify such portions of the Software'' which were ``in binary format'', this volates the freedom to ``change it... as you wish''
    \item Section 2: The user is not allowed to ``reverse engineer or decompile'' any ``software in binary format'', this does not grant the freedom to ``study how the program works'' or ``change it... as you wish''.
\end{itemize}
\subsection{Conflicts with 2}
\begin{itemize}
  \item ``You may not use or distribute this Software... in any form for commercial purposes''
  \item  ``You will distribute them under the same terms and conditions as in this license'' means that I may not be able to help my neighbour. (i.e if he/she needs to do something prohibited by the license)
\end{itemize}
\subsection{Conflicts with 3}
\begin{itemize}
  \item  ``distribute the modified Software solely for non-commercial academic purposes''
  \item  ``You may not... distribute... any derivative works for commercial purposes''
\end{itemize}

It's also worth mentioning that section 9 can terminate the license entirely, violating all 4 freedoms.

\section{Question 2}
\subsection{Part a)}
Alice clearly can not have a case against Charlie based on contract law. Charlie did not enter any agreement with Alice and even is we assume that Bob and Alice's contract is binding ``if
Alice enters into a contract with Bob, then Charlie is unrestricted by any of the obligations in that contract''. 

Hypothetically, if the supposed NDA between Alice and Bob is a binding contract then Alice could have a case. 
However, as established by Lord Dunenin in Dunlop Pneumatic Tyre Company, Ltd v Selfridge \& Co., Ltd a necessary condition for a contract is consideration from both sides (i.e a gift cannot be a contract).
Alice gives no considration and so Bob agreeing to anything is a gift. Even if the game were spelled correctly Alice would therefore have no case.

\subsection{Part b)}
 Copyright with regard to software is sufficient to ``protect the expression of an idea, but not the idea itself''.
 Alice therefore has no case in terms of copyright against either party regarding any ``ideas and principles which underlie any element of...[her] computer program''.
 As established by SAS Institute Inc. v World Programming Ltd ``clean room implementations in Europe are almost certainly legal''.
 Considering that Charlie has never seen the code, there is almost certainly no claim in terms of non-literal copying regardless of the similarity in functionality;
 we know that ``functionality is not protected by copyright'' as established by Navitaire Inc. v EasyJet Airline Co and another.

 Alice's only chance of a case in terms of copyright would be arguing in terms of literal copying based on Cantor Fitzgerald v Tradition.
 The method employed by Judge Pumfrey was to test whether the copied code represented a ``substantial part of the skill and labor of the author''.
 Considering that the code is released under GPL Alice would be able to look through the source code for any literal copying. Unfortunately, Charlie is a very skilled 
 programmer and so even if (in the very unlikely case) Bob could remember line for line exact implementations of tricky sections it is unlikely that this would constitute
 a significant part of Charlie's skill. 

\subsection{Part c)}
 Remedies would be only civil remedies only are applicable as Bob and Charlie are not attempting to make any money from this and so it is not a business context (and it is certainly not ``large scale'').
 The available options are damages, specific action or possibly an injunction if these are deemed insufficient

 Damages in terms of a contract breach means that the defendant has to pay a ``sum of money to the claimant sufficient to place the claimant in the same position he or she would have been in had the contract been correctly fulfilled''.
 This would entail determining her games' growth potential pre-release, which is a very difficult thing to do, perhaps by taking the average value of the business potential of similarly promising games?.
 Specific performance would not be an option as Bob can not retrospectively ``un-tell'' Charlie.
 If the damages are deemed too small then the Judge could grant an injuction. The scope of an injunction is very general and is an order to do or abstain from doing something. Since the code has already been released under the GPL it is unlikely
 that an injunction to stop Bob and Charlie from distributing the code would have any effect at this stage. 

\end{document}
